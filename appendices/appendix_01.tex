\chapter{Build and Deploy}\label{app:deploy}

	One benefit of using open source software and projects is, that one can benefit from continuous development of community
	and modern trends without any need to write extensive amount of code. Man just has to keep up and follow, what is
	happening and what is new.
	
	I consider Maven to be one of the best things that could ever happen to the Java development. A few years ago, when I
	was writing my bachelor's thesis, I remember setting up set of nested Ant \textit{build.xml} files, tons of properties
	and each modification took hours of studying the custom build process. Maven on the other hand follows the `convention
	instead of configuration` approach. This means, if not explicitly specified, all Maven based projects keep the same
	structure and share the same lifecycle, where build phases follow:

	\begin{itemize}
		\item \textbf{validate} the project is correct and all necessary information is available
		\item \textbf{compile} the source code of the project
		\item \textbf{test} the compiled source code using a suitable unit testing framework. These tests should not require
		the code be packaged or deployed 
		\item \textbf{package} takes the compiled code and packages it in its distributable format, such as a JAR, WAR, EAR
		\item \textbf{integration-test} - process and deploy the package if necessary into an environment where integration
		tests can be run
		\item \textbf{verify} - run any checks to verify the package is valid and meets quality criteria
		\item \textbf{install} - install the package into the local repository, for use as a dependency in other projects
		locally
		\item \textbf{deploy} - done in an integration or release environment, copies the final package to the remote
		repository for sharing with other developers and projects.
	\end{itemize}
	
	\section{Web application}
	
	RESTful API requires environment set up according to \ref{cha:implementation:platform}. Then follow
	database~installation~\ref{app:deploy:database}. After that, \textit{parent} and \textit{test-templates} need to be
	installed in a local Maven repository. To do so, checkout the two from my~GitHub~repositories~\cite{github} or use
	snapshots from the~attached~CD~\ref{app:CDcontent}:
	
	\begin{verbatim}
		parent > mvn install
		test-templates > mvn install
	\end{verbatim}

	When RESTful API is checkout fresh from its repository, then nothing should stop the packaging process:

	\begin{verbatim}
		MI-MPR-DIP > mvn package
	\end{verbatim}
	
	To make sure  nothing is left from some older build:
	
	\begin{verbatim}
		MI-MPR-DIP > mvn clean package
	\end{verbatim}
	
	This will compile all sources, execute both unit and integration tests and create deployable WAR package. This can be
	found in \textbf{target/admission.war}. This file can be deployed in any servlet container, such as Jetty or Tomcat.
	Application server is not required.
	
	RESTful API is also configured for simple deployment to Tomcat 7 container. In the \textbf{pom.xml} file, several
	properties should be edited:

	\lstset{language=XML}
	\begin{lstlisting}[tabsize=2]
	<maven.tomcat7.url>
		http://tomcat7-host:port/manager/html
	</maven.tomcat7.url>
	<maven.tomcat7.username>USERNAME</maven.tomcat7.username>
	<maven.tomcat7.password>PASSWORD</maven.tomcat7.password>
	\end{lstlisting}

	Now Maven is ready to deploy RESTful API into Tomcat 7:

	\begin{verbatim}
		MI-MPR-DIP > mvn clean package tomcat7:deploy
	\end{verbatim}
	
	\section{Database}\label{app:deploy:database}
	
	Because no application logic is written using SQL, i.e. no stored procedures nor triggers, database machine can be as
	simple and resource saving as possible. MySQL with InnoDB engine suits well.

	RESTful API source code is shipped with \textbf{.sql} schema installation file. This can be found in
	\textbf{src/main/resources/META-INF/sql/admission.sql}. Installation in MySQL follows:
	
	\begin{verbatim}
		~ > mysql
		mysql > create database admission character set utf8 collate utf8_bin;
		mysql > quit
		~ > mysql admission < src/main/resources/META-INF/sql/admission.sql 
	\end{verbatim}

	The last step is to point the RESTful API to the MySQL server. This is done via \textbf{.property} file, which can be
	found in \textbf{src/main/resources/META-INF/deployment.properties}. Concrete keys are:
	
	\begin{verbatim}
		database.url=jdbc:mysql://mysql-host:port/admission
		database.username=USERNAME
		database.password=PASSWORD
	\end{verbatim}
