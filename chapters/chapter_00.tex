\begin{introduction}
	\section{Motivation and objectives}

	Every year, hundreds of high school graduates apply for studies at Czech Technical University, Faculty of Informatics.
	This raises certain requirements, including managing, storing, analysing and processing of all these applications.
	Each application has its own life cycle, which begins with filling out an on-line form and continues through various
	steps which an applicant has to pass. The life cycle ends when a decision of acceptance is delivered to the applicant
	and he either enrolls in the studies or not.
	
	We live in the world of new era of the Internet. Everything goes on-line, web and the latest trend - everything goes
	mobile. People want things to happen very quickly. They want to access all the information fast, now.
	
	Students and applicants are no different. They expect from this prestigious University, especially from Faculty of
	Informatics, the most modern and useful gadgets when it comes to software and web.
	
	\section{How do things work now}
	
	Currently all applications are processed rather manually. Many man days of administrative work are consumed during the
	process. Although an electronic form is filled in and submitted by an applicant, the rest of actions almost exclusively
	fall into the hands of Study Department staff. Some of the work is handled by simple scripts or other utilities. The
	question is: Why don't we do most of the work automatically?
	
	This work is monotonous and can even lead to men's frustration.
	
	\section{What should be achieved - the goals}
	
	Courses at Faculty of Informatics teach its students to handle various programming languages, web technologies and
	techniques.
	We all know what to expect from a working web application and good looking one is a bonus. This is why knowledge of
	faculty's students should be used for good of their successors. Fast, reliable, informative and functional system will
	make them feel more comfortable and perhaps could even save some precious time.
	
	Ideal state would be to accept on-line applications and automatically generate invitations for applicants, that should
	attend a test. After the test, process all results and generate a decision of acceptance letter for all who passed the
	test or are accepted without it. The only manual interventions that will remain is to accept apology, appeal and insert
	the letters into the envelopes.

	Pragmatically, goals of this thesis could be summarized as follows:
	
	\begin{itemize}
	  \item familiarise with RESTful best practices, patterns and anti-patterns
	  \item familiarise with \gls{BPM} with main focus on jBPM
	  \item implement RESTful \gls{API} (back-end) according to functionality requested by Android and Web UI teams
	  \item implement admission processing using Java and jBPM processing machine
	  \item explore new and modern Java (JEE) technologies
	  \item follow modern development methodologies
	  \item perform tests during and after development
	  \item use exclusively Open Source software and tools
	\end{itemize}
	
	\section{Let's make things better}
	
	Taking the above written into account, this might be a good idea for a master's or bachelor's thesis. However if we
	want to use all available technologies that have become popular in past years and automatize the majority of admission
	processing, it turns out to be a very complex project. So why not to create several teams and split necessary work into
	multiple, both bachelor's and master's, thesis?
	
	This is how project Přiříz was born. It includes web interface for both students and Study Department staff, native
	Android application and RESTful \gls{API} with \gls{BPM} processing machine, which is the subject of my master's thesis.
	
	\section{Structure of this work}
	
	Basically, I could divide my work into these main parts, which are then further split into chapters:
	
	\begin{itemize}
	  \item Theoretical introduction is covered by chapters \ref{cha:rest} and \ref{cha:bpm}. Following parts will largely
	  draw on the information contained here.
	  \item Analytical part consists of chapters \ref{cha:architecture} and \ref{cha:technologies}. Talks about
	  architecture of the whole ecosystem with main focus on RESTful \gls{API}.
	  Describes technologies used, methodologies applied and tools commonly used during development and testing phases.
	  Theoretical and analytical parts together partially form Feasibility Study.
	  \item Implementation and testing (unit, integration and regression) is covered by chapters \ref{cha:implementation}
	  and \ref{cha:testing}. They form so called Detailed Design.
	  \item Results and conclusion is a final chapter of this work \ref{cha:conclusion}.
	\end{itemize}
	
	Appendices at the end of the document are referred directly from the text within the chapters. Smaller figures, tables
	or other objects are put directly into the content.
	 
\end{introduction}