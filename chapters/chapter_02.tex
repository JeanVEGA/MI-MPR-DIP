\chapter{RESTful API with JAX-RS}\label{rest}

	Nowadays, Internet consumers demand fast growth of various services and integration of their favourite ones. As en
	example I can point out synchronization of contact list between very popular social networks, e-mail providers and
	phone contact lists. 
	
	Other example may be growing amount of \verb|mashups|\footnote{Applications that are created via
	combination of multiple different services. Such application, almost exclusively web based, can be created very quickly
	by consuming several \gls{API}s. Not necessarily from the same provider.} and uncountable number of
	\verb|startups|\footnote{Constantly rising amount of web applications, that focus on fast growth of attracted users.
	They offer various services, which are often very innovative and experimental. One successful example is popular
	social network and my favorite information channel - Twitter.}, who often provide RESTful or different type of public
	\gls{API}.

	\section{Talking about REST, what is it?}
	
	\gls{REST} or RESTful programming is not defined by any official standard and there are no official guidelines or rules
	for it.
	So what is it then? It is and architectural and programming style for Web, where a set of constraints is defined. Lots
	of text has been written about it during past years and describing the whole idea of REST is out of scope of this
	master's thesis. I can however try to point out the most significant, important and basically, what I personally
	managed to adopt.
	
	\subsection{Main principles of REST, RESTful web service}
	
	There are several architectural principles, that one should keep in mind when thinking of REST \cite[p.~3]{restful}:
	
	\begin{itemize}
	  \item Addressable resources 
	  	The key abstraction of information and data in REST is a resource, and each re-
		source must be addressable via a \gls{URI}.
		A uniform, constrained interface
		Use a small set of well-defined methods to manipulate your resources.
		\item Representation-oriented 
		You interact with services using representations of that service. A resource refer-
		enced by one URI can have different formats. Different platforms need different
		formats. For example, browsers need HTML, JavaScript needs JSON (JavaScript
		Object Notation), and a Java application may need XML.
		\item Communicate statelessly
		Stateless applications are easier to scale.
		\gls{HATEOAS}
		Let your data formats drive state transitions in your applications.
	\end{itemize}
	
	\gls{HATEOAS} is often understood as a core principle of \gls{REST}. It carries an idea of resource representation via
	links and stateless implementation of services.
	
	RESTful web services are the result of applying these constraints to services that utilize web standards such as
	\gls{URI}s, \gls{HTTP}, \gls{XML}, and \gls{JSON}.
	
	\subsection{Back to the roots, HTTP is reborn}
	
	\gls{SOA} has been in this world for a long time. Many different approaches and technologies exist to implement it.
	From those worth to mention: DCE, CORBA, Java RMI, \ldots They offer robust standards, one can build large, complex
	and scalable systems on top of it, but there is a cost. They often bring huge complexity and maintenance requirements
	into place.
	
	Currently, when one says \gls{SOA}, it often evokes \gls{SOAP} in a mind, that spent several years using technologies
	mentioned above. This however is not a bad thing. \gls{SOAP} is used very widely and is perfectly suitable for
	developing services and \gls{API}s. But it is definitely not a lightweight technology and it is not ideal for
	everything. Its most common use case is for server-server communication in enterprise systems.
	
	Nowadays, we need something quickly adoptable, widely spreadable, platform and technology independent and client
	oriented. This needs a completely different approach and new way of thinking when it comes to \gls{SOA}. It is about
	Web, so why not to start with something that is Web, as we see it today, based on? Yes, it is \gls{HTTP}.
	
	Although \gls{REST} is not protocol specific, when saying \gls{REST} it usually automatically means \gls{REST} +
	\gls{HTTP}. No wonder. \gls{HTTP} is perfectly suitable for client-server \gls{SOA}, it is just about the way of
	thinking. It offers transport layer, request-response mechanism, descriptive responses, caching mechanism and many
	more. It is true, that in past years, when various types of web applications started to appear, many web developers
	limited their thinking and use of \gls{HTTP} to two basic cases:
	
	\begin{itemize}
	  \item GET a page with \gls{URI}, perhaps containing a few query parameters
	  \item POST a form
	\end{itemize}
	
	\begin{program}
	\caption{HTTP GET request/response example of a standard web page}\label{http_get_web}
	\begin{verbatim}
	GET /index.html HTTP/1.1
	User-Agent: curl/7.24.0 (i686-pc-cygwin) ...
	Host: www.google.sk
	Accept: text/html,application/xhtml+xml,application/xml;q=0.9,*/*;q=0.8
	Accept-Language: sk-sk,en;q=0.5

	HTTP/1.1 200 OK
	Date: Thu, 07 Jun 2012 11:25:15 GMT
	Expires: -1
	Cache-Control: private, max-age=0
	Content-Type: text/html; charset=ISO-8859-2
	Set-Cookie: ... expires= ...; path=/; domain=.google.sk
	Set-Cookie: ... expires= ...; path=/; domain=.google.sk; HttpOnly
	Server: gws
	X-XSS-Protection: 1; mode=block
	X-Frame-Options: SAMEORIGIN
	Transfer-Encoding: chunked

	<!doctype html><html ...><head> ... <body> ...
	\end{verbatim}
	\end{program}
	
	The example above \ref{http_get_web} shows most common HTTP request and response, when browsing the web via standard
	web browser. It requests object \textbf{/index.html} using \textbf{GET} method placed on host \textbf{www.google.sk}.
	My client also put several HTTP headers into the request:
	
	\begin{itemize}
	  \item \textbf{Accept:} text/html,application/xhtml+xml,application/xml;q=0.9,*/*;q=0.8
	  \item \textbf{Accept-Language:} sk-sk,en;q=0.5
	\end{itemize}
	
	Also the request does not contain any request body, as it is GETting information from the server. The response of the
	message received is 200, which means OK - success. An overview of all available HTTP response codes can be found
	on-line at \cite{httpcodes}.
	
	RESTful web service needs more than that and luckily \gls{HTTP} offers much more.
	
	\section{Why not SOAP?}
	
	\section{REST vs. SOAP}
	
	\section{REST, Java and JAX-RS}