\chapter{RESTful API with JAX-RS}\label{rest}

	Nowadays, Internet consumers demand fast growth of various services and integration of their favourite ones. As en
	example I can point out synchronization of contact list between very popular social networks, e-mail providers and
	phone contact lists. 
	
	Other example may be growing amount of \verb|mashups|\footnote{Applications that are created via
	combination of multiple different services. Such application, almost exclusively web based, can be created very quickly
	by consuming several \gls{API}s. Not necessarily from the same provider.} and uncountable number of
	\verb|startups|\footnote{Constantly rising amount of web applications, that focus on fast growth of attracted users.
	They offer various services, which are often very innovative and experimental. One successful example is popular
	social network and my favorite information channel - Twitter.}, who often provide RESTful or different type of public
	\gls{API}.

	\section{Talking about REST, what is it?}
	
	\gls{REST} or RESTful programming is no official standard and there are no official guidelines or rules for it. So what
	is it then? It is and architectural and programming style for Web. Lots of text has been written about it during past
	years and describing the whole idea of REST is out of scope of this master's thesis. I can however try to point out the
	most significant, important and basically, what I personally managed to adopt.
	
	There are several architectural principles, that one should keep in mind when thinking of REST \cite[p.~3]{restful}:
	
	\begin{itemize}
	  \item Addressable resources 
	  	The key abstraction of information and data in REST is a resource, and each re-
		source must be addressable via a \gls{URI}.
		A uniform, constrained interface
		Use a small set of well-defined methods to manipulate your resources.
		\item Representation-oriented 
		You interact with services using representations of that service. A resource refer-
		enced by one URI can have different formats. Different platforms need different
		formats. For example, browsers need HTML, JavaScript needs JSON (JavaScript
		Object Notation), and a Java application may need XML.
		\item Communicate statelessly
		Stateless applications are easier to scale.
		\gls{HATEOAS}
		Let your data formats drive state transitions in your applications.
	\end{itemize}
	
	\gls{HATEOAS} is often understood as a core principle of \gls{REST}. It carries an idea of resource representation via
	links and stateless implementation of services.
	
	\subsection{Back to the roots, HTTP is reborn}
	
	\section{Why not SOAP?}
	
	\section{REST vs. SOAP}
	
	\section{REST, Java and JAX-RS}