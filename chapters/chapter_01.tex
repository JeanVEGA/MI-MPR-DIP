\begin{introduction}
	\section{Motivation and objectives}

	Every year hundreds of students apply for studies at Czech Technical University, Faculty of informatics. This
	raises certain requirements including managing, storing, analysing and processing of all these applications.
	Each application has its own life cycle, which begins with filling out an on-line form and continues through various
	steps which an applicant has to pass. The life cycle ends when a decision of acceptance is delivered to the applicant
	and he either enrolls in the studies or not.
	
	Currently all applications are processed rather manually. Many man days of administrative work are consumed during the
	process. Although an electronic form is filled in and submitted by an applicant, the rest of actions almost exclusively
	fall into the hands of Study Department staff. The question is: Why?
	
	We live in the world of new era of the Internet. Everything goes on-line, web and the latest trend - everything goes
	mobile. People want things to happen very quickly. They want to access all the information fast, now.
	
	Students and applicants are no different. They expect from this prestigious University, especially from Faculty of
	informatics, most modern and useful gadgets when it comes to software and web.
	
	\section{Let's make things better}
	
	Taking the above written into account, this might be a good idea for a master's or bachelor's thesis. However if we
	want to use all available technologies that have become popular in past years and to automatize majority of admission
	processing, it becomes a very complex project. So why not to build several teams and split it into multiple, both
	bachelor's and master's, thesis?
	
	This is how project Přiríz was born. It includes web interface for both students and Study Department staff, native
	Android application and RESTful API with BPM processing machine, which is a subject of my master's thesis.
	 
\end{introduction}